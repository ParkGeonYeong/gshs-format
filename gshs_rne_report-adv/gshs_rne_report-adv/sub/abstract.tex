%\maketitle  % command to print the title page with above variables
\makecover  % command to print the title page with above variables

\setcounter{page}{1}
\renewcommand{\thepage}{\roman{page}}

%----------------------------------------------
%   Table of Contents (자동 작성됨)
%----------------------------------------------
\cleardoublepage
\addcontentsline{toc}{section}{Contents}
\setcounter{secnumdepth}{3} % organisational level that receives a numbers
\setcounter{tocdepth}{3}    % print table of contents for level 3
\baselineskip=2.2em
\tableofcontents


%----------------------------------------------
%     List of Figures/Tables (자동 작성됨)
%----------------------------------------------
\cleardoublepage
\clearpage
\listoffigures	% 그림 목록과 캡션을 출력한다. 만약 논문에 그림이 없다면 이 줄의 맨 앞에 %기호를 넣어서 코멘트 처리한다.

\cleardoublepage
\clearpage
\listoftables  % 표 목록과 캡션을 출력한다. 만약 논문에 표가 없다면 이 줄의 맨 앞에 %기호를 넣어서 코멘트 처리한다.


\cleardoublepage
\clearpage

%---------------------------------------------------------------------
%                  영문 초록을 입력하시오
%---------------------------------------------------------------------
%\begin{abstracts}     %this creates the heading for the abstract page
%	\addcontentsline{toc}{section}{Abstract}  %%% TOC에 표시
%	\noindent{
%			Put your abstract here. Once upon a time, \gshs said : `The first, and the best.'
%	}
%\end{abstracts}

%\cleardoublepage
%\clearpage

\begin{abstractskor}
	\addcontentsline{toc}{section}{초록}  %%% TOC에 표시
	\noindent{
		
	여기에 국문 초록을 입력하십시오
	
	}
\end{abstractskor}






