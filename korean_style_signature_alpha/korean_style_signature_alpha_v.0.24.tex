% 2017/01/11 v.0.24
% (인) 표시 위에 서명/날인 그림을 넣는 방법. 알파 버전.
% 양식을 가져와서 편집하였음.
% 서명 종횡비가 클 경우 크기를 더 축소.
% 우측으로 치우치는 문제에 대해 1mm 보정.
\documentclass[a4paper]{article}
\usepackage[showframe,top=35mm,bottom=30mm,left=30mm,right=30mm]{geometry}
% 여백을 보여주기 위해 showframe 옵션 사용.
\usepackage{graphicx}
\usepackage{tikz}
\usepackage{kotex}
\usepackage{adjustbox}
\usepackage{calc} %길이 계산을 위함.
\usepackage{ifthen}


\newenvironment{myepigraph}
{\par\hfill
	\begin{tabular}{@{}r@{\hspace{20mm}}}%
	}%
	{%
	\end{tabular}\par\medskip}

\def\apprvsignone{parkdaegam.png}	%첫 번째 서명 파일
\def\apprvsigntwo{ntieaK.png}		%두 번째 서명 파일
\def\apprvsignthree{hongpanseo.png}	%세 번째 서명 파일
\def\apprvsign{(인)}

\newlength{\apprvsignonewidth}
\newlength{\apprvsigntwowidth}
\newlength{\apprvsignthreewidth}
\newlength{\apprvsignoneheight}
\newlength{\apprvsigntwoheight}
\newlength{\apprvsignthreeheight}
\settowidth\apprvsignonewidth{\includegraphics[height=12mm]{\apprvsignone}}
\settowidth\apprvsigntwowidth{\includegraphics[height=12mm]{\apprvsigntwo}}
\settowidth\apprvsignthreewidth{\includegraphics[height=12mm]{\apprvsignthree}}
\setlength{\apprvsignoneheight}{12mm}
\setlength{\apprvsigntwoheight}{12mm}
\setlength{\apprvsignthreeheight}{12mm}
\newcommand{\apprvsignonevalid}{true}
\newcommand{\apprvsigntwovalid}{true}
\newcommand{\apprvsignthreevalid}{true}
\ifthenelse{\lengthtest{\apprvsignonewidth < 25mm}}{}{%
	\renewcommand{\apprvsignonevalid}{false}%
	\setlength{\apprvsignonewidth}{25mm}%
	\settoheight\apprvsignoneheight{%
		\includegraphics[width=25mm]{\apprvsignone}%
	}
}
\ifthenelse{\lengthtest{\apprvsigntwowidth < 25mm}}{}{%
	\renewcommand{\apprvsigntwovalid}{false}%
	\setlength{\apprvsigntwowidth}{25mm}%
	\settoheight\apprvsigntwoheight{%
		\includegraphics[width=25mm]{\apprvsigntwo}%
	}
}
\ifthenelse{\lengthtest{\apprvsignthreewidth < 25mm}}{}{%
	\renewcommand{\apprvsignthreevalid}{false}%
	\setlength{\apprvsignthreewidth}{25mm}%
	\settoheight\apprvsignthreeheight{%
		\includegraphics[width=25mm]{\apprvsignthree}%
	}
}
\newlength{\apprvsignwidth}
\settowidth\apprvsignwidth{\apprvsign}

\begin{document}
	
	\vspace*{3em}
	
	\begin{myepigraph}
		{\Large {몇년몇월며칠}}\\[11mm]
		{\Large 
		심사위원장\hspace{5mm}박 대 감\hspace{6mm}%
		\trimbox{%
			{4.4mm} {\apprvsignoneheight/2} %
			{\apprvsignonewidth-\apprvsignwidth-1mm-2.67mm} %
			{\apprvsignoneheight/2}}
		{%
			\begin{tikzpicture}
			\draw (\apprvsignonewidth/2-\apprvsignwidth-1mm,0) 
			node {%
				\includegraphics[height=12mm]{\apprvsignone}};
				\draw (0,0) node {\apprvsign};
			\end{tikzpicture}}%
		}\\[8mm]
		{\Large 
		심사위원\hspace{5mm}김 교 수\hspace{6mm}%
		\trimbox{%
			{4.4mm} {\apprvsigntwoheight/2} %
			{\apprvsigntwowidth-\apprvsignwidth-1mm-2.67mm} %
			{\apprvsigntwoheight/2}}
		{%
			\begin{tikzpicture}
			\draw (\apprvsigntwowidth/2-\apprvsignwidth-1mm,0) 
			node {%	
				\includegraphics[height=\apprvsigntwoheight]%
				{\apprvsigntwo}%
			};
				\draw (0,0) node {\apprvsign};
			\end{tikzpicture}}%
		}\\[8mm]
		{\Large 
		심사위원\hspace{5mm}홍 판 서\hspace{6mm}%
		\trimbox{%
			{4.4mm} {\apprvsignthreeheight/2} %
			{\apprvsignthreewidth-\apprvsignwidth-1mm-2.67mm} %
			{\apprvsignthreeheight/2}}
		{%
			\begin{tikzpicture}
			\draw (\apprvsignthreewidth/2-\apprvsignwidth-1mm,0) 
			node {%			
				\includegraphics[height=12mm]{\apprvsignthree}};
				\draw (0,0) node {\apprvsign};
			\end{tikzpicture}}%
		}
	\end{myepigraph}
	% 여기까지 수정한 양식.


	\vspace{3em}

	\begin{myepigraph}
	{\Large {몇년몇월며칠}}\\[11mm]
	{\Large 심사위원장\hspace{5mm}박 대 감\hspace{6mm}(인)}\\[8mm]
	{\Large 심사위원\hspace{5mm}김 교 수\hspace{6mm}(인)}\\[8mm]
	{\Large 심사위원\hspace{5mm}홍 판 서\hspace{6mm}(인)}
	\end{myepigraph}
	% 여기까지 원래 양식. 간격과 배치를 보기 위한 대조본으로 실제 코드의 
	% 커맨드들을 글자로 치환함.
\end{document}