\documentclass[11pt]{article}
\usepackage[a4paper]{geometry} 
\usepackage{kotex} %to use hangul
\usepackage{endnotes} %to use endnotes
\usepackage[group-separator={,}]{siunitx} %si unit
\usepackage{tocloft} %adjusting overlaps
\usepackage{ragged2e} %alignment (right)
\usepackage{setspace} %double spacing
\usepackage{graphicx}
\usepackage{dcolumn} %decimal point alignment
\usepackage{longtable}
\usepackage{booktabs}
\usepackage{caption}
\usepackage{indentfirst}
\begin{document}
	\setlength\parindent{15pt}
	%endnote format
	\renewcommand\makeenmark{\textsuperscript{[\theenmark]}}
	
	%section numbering
	\renewcommand{\thesection}{\arabic{section}}
	\renewcommand{\thesubsubsection}{\thesubsection.\arabic{subsubsection}}
	
	%table numbering
	%\renewcommand{\thetable}{\arabic{table}}
	
	\cftsetindents{section}{0em}{2em}
	\cftsetindents{subsection}{1em}{2em}
	%\cftsetindents{table}{4em}{5em}
	
	%table decimal point alignment
	\newcolumntype{d}[1]{D{.}{.}{#1}}
	
	%use comma, not dots
	%\sisetup{locale = FR}
	\sisetup{
		input-ignore={,},
		group-digits=true,
		group-separator = \text{\,}}
	
	%begin template
	\oddsidemargin = 42pt
	\topmargin = 0pt
	\headheight = 0pt
	\headsep = 25pt
	\textheight = 569pt
	\textwidth = 384pt
	\marginparsep = 10pt
	\marginparwidth = 35pt
	\footskip = 56pt
	\marginparpush = 7pt
	\hoffset = 0pt
	\voffset = 0pt
	\paperwidth = 597pt
	\paperheight = 845pt
	%end template
	
	\doublespacing
	
	%page numbering I
	\pagenumbering{roman}
	
	
	%constants
	\newcommand{\ttitle}{English Title} 
	\newcommand{\ttitleko}{한글 제목}
	
	
	\newcommand{\abstracten}{In this study,}
	\newcommand{\abstractko}{본 연구는 .}
	
	%내용: 연구 procedure(flow chart)
	%figure별 제목-설명 (아래, table 제목 위)
	%endnote 양
	
	
	%roman numbering for sections
	\renewcommand\thesection{\Roman{section}.}
	\renewcommand\thesubsection{\arabic{subsection}.}
	
	\begin{center}
		%\vspace{70mm}
		\vspace*{16mm}
		\Large{졸업논문청구논문}
		\\
		\vspace{15mm}
		\huge{\textbf{\ttitleko}}
		\\
		\vspace{7mm}
		\LARGE{\ttitle}
		\\
		\vspace{30mm}
		\LARGE{\textbf{홍길동 (洪 吉 東, Hong, Gil Dong)\\14000}}
		\\
		\vspace{12mm}
		\LARGE{\textbf{과학영재학교 경기과학고등학교}}
		\\
		\vspace{7mm}
		\LARGE{\textbf{2017}}
	\end{center}
	
	\newpage
	
	\begin{center}
		\vspace*{30mm}
		\huge{\ttitleko}
		\\
		\vspace{10mm}
		\LARGE{\ttitle}
	\end{center}
	
	\vspace{30mm}
	\begin{center}
		[논문제출 전 체크리스트]
	\end{center}
	
	\small
	1. 이 논문은 내가 직접 연구하고 작성한 것이다. ■
	
	2. 인용한 모든 자료(책∙논문∙인터넷자료 등)의 인용표시를 바르게 하였다. ■
	
	3. 인용한 자료의 표현이나 내용을 왜곡하지 않았다. ■
	
	4. 정확한 출처제시 없이 다른 사람의 글이나 아이디어를 가져오지 않았다. ■
	
	5. 논문 작성 중 도표나 데이터를 조작(위조 혹은 변조)하지 않았다. ■
	
	6. 다른 친구와 같은 내용의 논문을 제출하지 않았다. ■
	\normalsize
	
	\newpage
	
	\begin{center}
		\LARGE{\ttitle}
		\vspace{20mm}
		
		\Large
		Advisor: Teacher Hong, Pan-Seo
		
		by
		\vspace{5mm}
		
		\textbf{14000 Hong, Gil-Dong}
		
		\textbf{Gyeonggi Science Highschool for the Gifted}
	\end{center}
	
	\vspace{10mm}
	A thesis submitted to the Gyeonggi Science Highschool in partial fulfillment of the requirements for the graduation. The study was conducted in accordance with Code of Research Ethics.
	\footnote{Declaration of Ethical Conduct in Research: I, as a graduate student of GSHS, hereby declare that I have not committed any acts that may damage the credibility of my research. These include, but are not limited to: falsification, thesis written by someone else, distortion of research findings or plagiarism. I affirm that my thesis contains honest conclusions based on my own careful research under the guidance of my thesis advisor.}
	
	\vspace{7mm}
	\begin{flushright}
		2015. 6. 19.
	\end{flushright}
	
	\setlength{\parindent}{60mm}
	\Large
	\textbf{Approved by}
	
	\textbf{Teacher Hong, Pan-Seo}
	
	\textbf{[Thesis Advisor]}
	\setlength{\parindent}{0mm}
	
	
	\setlength\parindent{15pt}
	
	\newpage
	
	\begin{center}
		\huge{\ttitleko}
		
		\vspace{20mm}
		\LARGE{홍길동}
		
		\vspace{30mm}
		\Large
		위 논문은 과학영재학교 경기과학고등학교 졸업논문으로 졸업논문심사위원회에서 심사 통과하였음.
		
	\end{center}
	\vspace{20mm}
	\begin{flushright}
		2015년 6월 25일
		
		\vspace{10mm}
		심사위원장 홍판서 (인)
		
		\vspace{10mm}
		심사위원 박완규 (인)
		
		\vspace{10mm}
		심사위원 선병호 (인)
		
	\end{flushright}
	
	\newcommand{\englishabs}{
		\newpage
		\begin{center}
			\begin{LARGE}
				\ttitle
			\end{LARGE}
		\end{center}
		\vspace{40mm}
		
		\begin{center}
			\LARGE{\textbf{Abstract}}
		\end{center}
		
		\vspace*{15mm}
		
		\abstracten
	}
	
	\addcontentsline{toc}{section}{Abstract}
	
	\englishabs
	
	\newcommand{\koreanabs}{
		\newpage
		
		\begin{center}
			\begin{LARGE}
				\ttitleko
			\end{LARGE}
		\end{center}
		\vspace{40mm}
		
		\begin{center}
			\LARGE{\textbf{초 록}}
		\end{center}
		
		\vspace*{15mm}
		
		\abstractko
	}
	
	\addcontentsline{toc}{section}{초록}
	
	\koreanabs
	
	\newpage
	\addcontentsline{toc}{section}{Contents}
	
	\setcounter{tocdepth}{3}
	\renewcommand{\cftsecleader}{\cftdotfill{\cftdotsep}}
	
	\tableofcontents
	
	\newpage
	
	\addcontentsline{toc}{section}{List of Tables}
	\listoftables
	
	\newpage
	
	\addcontentsline{toc}{section}{List of Figures}
	\listoffigures
	
	\newpage
	
	%page numbering 2
	\pagenumbering{arabic}
	\setcounter{page}{1}
	

	
	\section{Introduction}
	Fo	
	\section{Result}
	\cite{asdf}
	\newpage
	\addcontentsline{toc}{section}{References}
	\begin{thebibliography}{00}
		\bibitem{asdf}{gshslatexintro(2015),\textbf{0},0-0}
		% google scholar에서 '인용'을 누르면 MLA 형식과 APA 형식이 모두 나옵니다. 그것을 두번째 괄호 안에 그대로 복붙하면 됩니다.
	\end{thebibliography}
	
	
	\newpage
	\begin{center}
		\begin{huge}
			\textbf{Summary}
			
			\ttitle
		\end{huge}
	\end{center}
	asdf
	
	\newpage
	\addcontentsline{toc}{section}{감사의 글}
	\begin{center}
		\begin{LARGE}
			\textbf{감사의 글}
		\end{LARGE}
	\end{center}
	 정말 감사합니다.
	
	\newpage
	\addcontentsline{toc}{section}{연구 활동}
	\begin{center}
		\begin{LARGE}
			\textbf{연구 활동}
		\end{LARGE}
	\end{center}
	상 받고싶다.
	
	
	
\end{document}