%%%%%%%%%%%%%%%%%%%%%%%%%%%%%%%%%%%%%%%
%%% Research note - Entry
%%% Template by Mikhail Klassen, April 2013
%%% 
%%%%%%%%%%%%%%%%%%%%%%%%%%%%%%%%%%%%%%%

%경기과학고 졸업논문, 알앤이 보고서에서 쓰던 클래스와 적절히 합침.  by 박기현(2017. 6. 5)
\documentclass{research-note-v1.0}

% 경기과학고 로고를 다음과 같이 넣어두세요."./img/logo.png"  by 박기현(2017. 6. 5)

\researchType{창의연구 R\&E  } % 기초 / 심화  /  창의연구 R\&E
%\reporttype{중간} % 중간 / 결과

\title{MODIS 자료를 이용한 한반도 주변의 에어로졸의 광학적 두께(AOD) 산출} % 제목 개행 시 \linebreak 사용. \\나 \newline 은 안됨.
\englishTitle{AOD Calculation around the Korean Peninsula using MODIS data}% 제목 개행 시 \linebreak 사용. \\나 \newline 은 안됨.

\author[1] {김재홍} % 제 1 저자명
\email[1]{@e-mail.address} % 제 1 저자 이메일
\author[2] {박준호} % 제 2 저자명
\email[2]{@e-mail.address} % 제 2 저자 이메일
\author[3] {윤산울} % 제 3 저자명
\email[3]{iu@bogo.sipda} % 제 3 저자 이메일
\author[4] {김병현} % 제 4 저자명
\email[4]{iu@bogo.sipda} % 제 4 저자 이메일
\advisor{박기현} % 지도교사명
\advisorEmail{guitar79@naver.com} % 지도교사 이메일

%\newcommand{\studentName}{연구자 : 김병현 외} %수정하세요.
\newcommand{\institution}{경기과학고등학교} %수정하세요.

%%%%%%%%%%%%%%%%%%%%%%%%%%%%%%%%%%%%%%%%%%%%%%%
%%%% researchtype이 '심화'일 경우에만 나타남 %%%%
\professor{교수님} % 지도교수명
\professorEmail{professor@e-mail.address} % 지도교수 이메일
%%%%%%%%%%%%%%%%%%%%%%%%%%%%%%%%%%%%%%%%%%%%%%%%
\summitdate{2017}{10}{20} % 제출일 (연, 월, 일)
\newtheorem{definition}{정의}

%\linespread{1.6}

\begin{document}
%\makecover


%\maketitle  % command to print the title page with above variables
\makecover   % command to print the title page with above variables


\pagenumbering{roman}                        % 로마자 페이지 시작
\setcounter{page}{1}
%---------------------------------------------------------------------
%                  영문 초록을 입력하시오
%---------------------------------------------------------------------
%\begin{abstracts}     %this creates the heading for the abstract page
%	\addcontentsline{toc}{section}{Abstract}  %%% TOC에 표시
%	\noindent{
%		Put your abstract here. 
%	}
%\end{abstracts}
%
%---------------------------------------------------------------------
%                  국문 초록을 입력하시오
%---------------------------------------------------------------------
%\begin{abstractskor}        %this creates the heading for the abstract page
%	\addcontentsline{toc}{section}{초록}  %%% TOC에 표시
%	\noindent{
%		초록
%	}
%\end{abstractskor}


%----------------------------------------------
%   Table of Contents (자동 작성됨)
%----------------------------------------------
\cleardoublepage
\addcontentsline{toc}{section}{Contents}
\setcounter{secnumdepth}{3} % organisational level that receives a numbers
\setcounter{tocdepth}{3}    % print table of contents for level 3
\baselineskip=2.2em
\tableofcontents


%----------------------------------------------
%     List of Figures/Tables (자동 작성됨)
%----------------------------------------------
\cleardoublepage
\clearpage
\listoftables
% 표 목록과 캡션을 출력한다. 만약 논문에 표가 없다면 이 위 줄의 맨 앞에 
% `%' 기호를 넣어서 주석 처리한다.

\cleardoublepage
\clearpage
\listoffigures
% 그림 목록과 캡션을 출력한다. 만약 논문에 그림이 없다면 이 위 줄의 맨 앞에 
% `%' 기호를 넣어서 주석 처리한다.

\cleardoublepage
\clearpage
\renewcommand{\thepage}{\arabic{page}}
\setcounter{page}{1} % Abstract
		
\section*{May, 31, 2017}

\begin{itemize}
	\item{연구 장소 : 과학영재연구센터 627호}
	\item{참여자 : {\@firstAuthor}, \@secondAuthor, \@thirdAuthor, \@fourthAuthor}
\end{itemize}

	\begin{table}[h]
	%	\caption{Research Note.}
		\label{table01}
%		\centering
		\begin{tabular}{|c|c|} 
			\hline
			연구 장소   &   과학영재연구센터 627호 \\ 
			\hline
			참여자   &   {\@firstAuthor}, \@secondAuthor, \@thirdAuthor, \@fourthAuthor  \\
			\hline
		\end{tabular}
	\end{table}

	\subsection*{연구 계획 수립}
							연구 계획을 수립하기 위하여 만남.
							연구 계획을 수립하기 위하여 만남.
								연구 계획을 수립하기 위하여 만남.
									연구 계획을 수립하기 위하여 만남.
									v
	
    \subsection*{역할 분담}
			연구 계획을 수립하기 위하여 만남.
				연구 계획을 수립하기 위하여 만남.
					연구 계획을 수립하기 위하여 만남.
						연구 계획을 수립하기 위하여 만남.
							연구 계획을 수립하기 위하여 만남.
								연구 계획을 수립하기 위하여 만남.
									연구 계획을 수립하기 위하여 만남.
										연구 계획을 수립하기 위하여 만남.
	\subsection*{컴퓨터 셋업}
		연구 계획을 수립하기 위하여 만남.
			연구 계획을 수립하기 위하여 만남.
				연구 계획을 수립하기 위하여 만남.
					연구 계획을 수립하기 위하여 만남.
						연구 계획을 수립하기 위하여 만남.
							연구 계획을 수립하기 위하여 만남.

	\subsection*{Collection Data}
     Data는 다음과 같은 방법으로 order하여 다운로드 받는다.

	\begin{itemize}
		\item{연구 주제의 전반적 관심을 조명.}
		\item{연구 분야의 스페셜 이슈를 조명.}
		\item{해당 이슈를 해결하기 위한 다양한 선행 연구들을 서술.}
		\item{선행 연구들의 한계점을 기술.}
		\item{한계를 극복하기 위한 본 연구의 목적을 밝힘.}
		\item{논문의 구성을 서술 (optional).}
	\end{itemize}


\section*{June, 5, 2017}

\subsection*{선행연구 고찰}

MODIS 관련 연구
\begin{itemize}
	\item{Characteristics of distribution and seasonal variation of aerosol optical depth in eastern China with MODIS products. \cite{ahn2012development}
	\item{Validation of MODIS aerosol optical depth retrieval over land. \cite{chu2002validation}
\end{itemize}

\subsection*{분석}

 분석 내용 작성

\subsection*{회의}

회의 내용 작성

\subsection*{결과}

결과 작성


\subsection*{다음 계획}


\section*{May, 31, 2016}

\subsection*{실험}

실험 내용 작성

\subsection*{분석}

분석 내용 작성

\subsection*{회의}

회의 내용 작성

\subsection*{결과}

결과 작성


\subsection*{다음 계획}

\bibliography{research-note}
\bibliographystyle{plain}

\end{document}