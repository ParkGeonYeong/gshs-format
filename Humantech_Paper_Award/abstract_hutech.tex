\documentclass{abstract_hutech}

\begin{document}
\thispagestyle{firstpage}
\twocolumn[
\begin{@twocolumnfalse}
\begin{center}
\fontsize{11}{0}\selectfont{``Please evaluate yourself with the following evaluation criteria''}
\vspace{1.5em}\par
\fontsize{16}{0}\selectfont{\textbf{Evaluation Criteria \& Self Evaluation}}
\end{center}
\vspace{10pt}
\fontsize{11}{0}\selectfont{\bf 1. Motivation \& Importance}\par\setlength\parindent{9pt}
\fontsize{10}{14}\selectfont{- Are the conclusions of this paper important considering the initial motivation of selecting this paper?}\vspace{9pt}\par
\noindent\fbox{
\parbox{\textwidth}{bla bla bla bla bla bla bla bla bla bla bla bla bla bla bla bla bla bla bla bla bla bla bla bla bla bla bla bla bla bla bla bla bla bla bla bla bla bla bla bla bla bla bla bla bla bla bla bla bla bla bla bla bla bla bla bla bla bla bla bla bla bla bla bla bla bla bla bla bla bla bla bla bla bla bla bla bla bla bla bla bla bla bla bla bla bla bla bla bla bla bla bla}
}
\vspace{20pt}\par\noindent
\fontsize{11}{0}\selectfont{\bf 2. Creativity}\par\setlength\parindent{9pt}
\fontsize{10}{14}\selectfont{- Does the paper suggest an interesting and creative idea?\par- Does the paper point out differences or new approaches from related researches?}\vspace{9pt}\par
\noindent\fbox{
\parbox{\textwidth}{bla bla bla bla bla bla bla bla bla bla bla bla bla bla bla bla bla bla bla bla bla bla bla bla bla bla bla bla bla bla bla bla bla bla bla bla bla bla bla bla bla bla bla bla bla bla bla bla bla bla bla bla bla bla bla bla bla bla bla bla bla bla bla bla bla bla bla bla bla bla bla bla bla bla bla bla bla bla bla bla bla bla bla bla bla bla bla bla bla bla bla bla}
}
\vspace{20pt}\par\noindent
\fontsize{11}{0}\selectfont{\bf 3. Evidences}\par\setlength\parindent{9pt}
\fontsize{10}{14}\selectfont{- Does the paper back up a theoretical idea, conclusion, analysis in the literature with sufficient experimental evidences?\par- Are the conclusions logically integrate with the issue and consistent?}\vspace{9pt}\par
\noindent\fbox{
\parbox{\textwidth}{bla bla bla bla bla bla bla bla bla bla bla bla bla bla bla bla bla bla bla bla bla bla bla bla bla bla bla bla bla bla bla bla bla bla bla bla bla bla bla bla bla bla bla bla bla bla bla bla bla bla bla bla bla bla bla bla bla bla bla bla bla bla bla bla bla bla bla bla bla bla bla bla bla bla bla bla bla bla bla bla bla bla bla bla bla bla bla bla bla bla bla bla}
}
\vspace{20pt}\par\noindent
\fontsize{11}{0}\selectfont{\bf 4. Practicality}\par\setlength\parindent{9pt}
\fontsize{10}{14}\selectfont{- Are topics and content of the practical possibilities?}\vspace{9pt}\par
\noindent\fbox{
\parbox{\textwidth}{bla bla bla bla bla bla bla bla bla bla bla bla bla bla bla bla bla bla bla bla bla bla bla bla bla bla bla bla bla bla bla bla bla bla bla bla bla bla bla bla bla bla bla bla bla bla bla bla bla bla bla bla bla bla bla bla bla bla bla bla bla bla bla bla bla bla bla bla bla bla bla bla bla bla bla bla bla bla bla bla bla bla bla bla bla bla bla bla bla bla bla bla}
}
\vspace{20pt}\par\noindent
\fontsize{11}{0}\selectfont{\bf 5. Possibility of future growth}\par\setlength\parindent{9pt}
\fontsize{10}{14}\selectfont{- Does the conclusions, implications, or outcomes of the paper make some contribution to either the application or\par the solution itself in the relevant field?}\vspace{9pt}\par
\noindent\fbox{
\parbox{\textwidth}{bla bla bla bla bla bla bla bla bla bla bla bla bla bla bla bla bla bla bla bla bla bla bla bla bla bla bla bla bla bla bla bla bla bla bla bla bla bla bla bla bla bla bla bla bla bla bla bla bla bla bla bla bla bla bla bla bla bla bla bla bla bla bla bla bla bla bla bla bla bla bla bla bla bla bla bla bla bla bla bla bla bla bla bla bla bla bla bla bla bla bla bla}
}

\end{@twocolumnfalse}
]

\newpage
\clearpage
\setcounter{page}{1}
\thispagestyle{firstpage}
\twocolumn[
\begin{@twocolumnfalse}
\vspace*{20pt}
\begin{flushleft}
\fontsize{20}{0}\selectfont{\textbf{Title}}
\vspace{32pt}\par
\fontsize{10}{12}\selectfont{\textbf{(Abstract) This is a template of Extended Abstract for HumanTech Paper Award. The recommended volume is 2 pages with 2-column format. Titles do not exceed two lines and abstracts do not exceed 15 lines.\\
Papers should be written in Times New Roman font with the font size of 20pt in bold for the title, 10pt in bold for abstracts, 11pt in bold for the titles within text, 10pt for the text, 9pt in bold for the titles of figures and tables and 9pt for the references. For the fairness of the review, Name, major, the school/university name, the school/university logo, and teacher/professors name of author should not be included in the abstract and paper.
}}
\end{flushleft}
\vspace{20pt}
\end{@twocolumnfalse}
]

\section{INTRODUCTION}
HumanTech Paper Award was established in 1994 with the purpose of encouraging Korean students to do research in science and technology. High school students and university students (Undergraduate \& Graduate) having Korean nationality and foreign student attending universities in Korea are eligible to submit papers to the HumanTech Paper Award. Paper should not be published any journal including online prior to full paper submission. HumanTech Paper Award has three stages of evaluation to select the awardees. The first stage will be done with an extended abstract, the second, with a full paper and the third, with an oral presentation. The submitted abstracts will be screened and the writers of the selected will be required to submit a full paper. Reviewers will be experts in each field. The objectives, scope, results, importance, and originality of the study should be described in the submitted abstracts.

\section{WRITING STYLE}
Abstracts and Papers should be written in A4 sized paper(21cm$\times$29.7cm) with margins of 3cm on the top, 2.5cm on the bottom, 1.5cm on the left and right, 2cm on the header, and 1cm on the footer.

``22nd HumanTech Paper Award'' should be written in the header. The header of first page is 12pt (bold), others are 9pt. The headers of even number pages should be left-justified, and those of odd number pages should be right-justified. Page numbers should be written in the footer. The footer of even number pages should be left-justified, and those of odd number pages should be right-justified.

The recommended volume is 2 pages with 2-column format. Titles do not exceed two lines and abstracts do not exceed 15 lines.

Papers should be written in English or Korean. Papers should be written in Times New Roman font for English, '바탕체' for Korean with the font size of 20pt in bold for the title, 10pt in bold for abstracts, 11pt in bold for the titles within text, 10pt for the text, 9pt in bold for the titles of figures and tables and 9pt for the references.

For the fairness of the review, Name, major, the school/university name, the school/university logo, and teacher/professors name of author should not be included in the abstract and paper.

The main text page should be divided into two columns vertically with the margin of 0.5cm between the two columns. If you insert titles, tables, graphs, or formulas in the main text, please insert one blank line before and after them.

The numbering of contents in the main text should use the following format: chapters are 1. , 2. , 3. et al., and paragraphs are 1.1. , 2.1. et al.

List and number all bibliographical references at the end of the document. When referring to them in the text, type the corresponding reference numbers in square brackets \cite{True00}.

The titles of tables, graphs, and pictures should be started with a capital letter. Tables’ titles should be written in the top of tables, pictures’ titles should be written in the bottom of pictures.

\begin{figure}[t]
\begin{center}
\includegraphics[width=6.23cm]{example-image-a}
\end{center}
\caption{\bf HumanTech Logo}\label{Fig01}
\end{figure}

Only SI units should be used and abbreviations should be spelled out when they appear first in the text. If a non-standard abbreviation is used first, it should be clearly defined.

References should be written to the following formats:

\noindent Authors should be listed surname first, followed by a comma and initials of given names. Titles of articles cited in reference lists should be in upright, not italic text; the first word of the title is capitalized, the title written exactly as it appears in the work cited, ending with a full stop. Book titles are italic with all main words capitalized. Journal titles are italic and abbreviated according to common usage. Volume numbers are bold. The publisher and city of publication are required for books cited. References to web-only journals should give authors, article title and journal name as above, followed by URL in full or DOI if known. References to websites should give authors if known, title of cited page, URL in full, and year of posting in parentheses. The year of publication (posting) should written in parentheses.

\section{CONCLUSION}

Abstracts and Papers should be written according to the following order: ①Title ②Abstract ③Main text ④References.

\begin{thebibliography}{99}
\bibitem{True00} True, H. L. and Lindquist, S. L. A. A yeast prion provides a mechanism for genetic variation and phenotypic diversity. {\it Nature} {\bf 407}, 477--483 (2000)
\bibitem{Schluter00} Schluter, D. {\it The Ecology of Adaptive Radiation} (Oxford Univ. Press, 2000)
\bibitem{Plazzo11} Plazzo, A. P. et al. Bioinformation and mutational analysis of channelrhodopsin-2 cation conducting pathway. {\it J. Biol. Chem.} http://dx.doi.org/10.1074/jbc.M111.326207 (2011)
\end{thebibliography}







\end{document}