\section{서론}
%\section{Introduction}

\subsection{BibTeX}

BibTeX을 사용하면 참고문헌을 쉽게 정리하거나 관리할 수 있다. 다음은 Google 학술검색 사이트에서 찾은 논문에서 BibTeX 코드를 찾는 방법이다.

\begin{enumerate}
	\item Google 학술검색에서 원하는 논문을 찾는다.
	\item 인용 버튼을 누르면 MLA, APA 등 다양한 인용 스타일이 나온다. 여기서 창 하단의 BibTeX을 클릭한다.
	\item BibTeX을 클릭해서 나온 코드 전문을 복사하여 보고서에 있는 폴더의 bibfile.bib 파일에 붙여넣기 한다.
\end{enumerate}

다음은 예시로 찾은 논문의 BibTeX 코드이다.
\begin{lstlisting}
@article{ward2001landslide,
title={Landslide tsunami},
author={Ward, Steven N},
journal={Journal of Geophysical Research: Solid Earth},
volume={106},
number={B6},
pages={11201--11215},
year={2001},
publisher={Wiley Online Library}
}
\end{lstlisting}

BibTeX에 넣은 논문을 인용하기 위해서는 cite 명령어를 사용한다. 위 코드에서 @article 오른쪽에 있는 문구가 ward2001landslide이므로 이 논문을 인용하기 위해서 명령어
\begin{lstlisting}
\cite{ward2001landslide}
\end{lstlisting}
를 사용한다. 이 명령어 실행 시 명령어를 작성한 자리에 \cite{ward2001landslide}과 같이 논문을 인용했다는 표시가 뜬다. 인용한 논문의 목록은 문서의 References에 표시된다.