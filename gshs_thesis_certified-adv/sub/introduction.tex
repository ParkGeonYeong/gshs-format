\section{Introduction}

서론은 연구를 진행하게 된 배경을 기술하는 곳으로 보통 다음과 같은 순서로 쓰는 편이다.
\begin{itemize}
\item{연구 주제의 전반적 관심을 조명.}
\item{연구 분야의 스페셜 이슈를 조명.}
\item{해당 이슈를 해결하기 위한 다양한 선행 연구들을 서술.}
\item{선행 연구들의 한계점을 기술.}
\item{한계를 극복하기 위한 본 연구의 목적을 밝힘.}
\item{논문의 구성을 서술 (optional).}
\end{itemize}
서론은 과거부터 현재까지 해당 분야의 연구 진행을 기술하기 때문에 선행 연구 논문들을 레퍼런스로 도입하는 경우가 빈번하게 나타난다. \LaTeX 에서 참고문헌을 표기하는 방법을 알아보자. 먼저 이 문서의 후반부에 위치한 레퍼런스 부분을 찾아간다. 이 문서를 컴파일했을 때 생성된 PDF 파일에는 {\bf References}라고 나와 있지만 여기서는 {\textbackslash}begin\{thebibliography\}\{99\}로 시작에서 {\textbackslash}end\{thebibliography\}로 종료되는 그 사이에 참고문헌을 작성하면 된다. 여기서 숫자 99는 참고문헌이 100개 넘는 논문을 작성하는 것이 아니라면 그대로 놔둔다. 참고문헌 작성 예시는 다음과 같다.
\begin{lstlisting}
\bibitem{Mok06}C. Mok, C.-M. Ryu, P. H. Yoon, and A. T. Y. Lui, ``Global twofluid stability of bifurcated current sheets'', J. Goephys. Res., {\bf 111}, A03203
(2006).
\end{lstlisting}
{\textbackslash}bibitem 다음의 \{ \} 안에는 자신이 그 논문을 기억하기 쉬운 규칙을 정하여 작성하면 된다. 보통 논문 주저자의 last name과 논문 출판 년도를 사용하여 표기한다. 그리고 저자들, 논문 제목, 저널 이름, 권, 호, 페이지, 출판년도 순으로 입력한다. 저자는 3인 이하일 경우에는 모두 적도록 하고 4인 이상일 때는 주저자만 작성하고 그 외는 et al.이란 표기로 대체한다. 논문 제목은 큰 따옴표로 묶어준다. \LaTeX에서 시작하는 따옴표는 키보드에서 숫자 1 왼쪽 버튼, 마치는 따옴표는 키보드 엔터키 왼쪽 버튼을 사용함에 유의하라. 저널 이름은 경우에 따라 약어를 사용할 수 있다. 약어를 사용할 때는 정식으로 정해진 약어임을 확인한 후 사용한다. Volume(권)은 굵은 글자 처리한다. 위의 예시처럼 {\textbackslash}bf를 사용하면 된다. Number(호)는 경우에 따라 없는 저널도 있다. 위의 예시에는 `호'를 넣지 않았다. 만약 `호'를 넣고자 할 때는 둥근 괄호로 묶어준다. 마지막으로 페이지와 출판년도를 작성한다. 출판년도는 둥근 괄호로 묶어준다.

이제 서론에서 해당 논문을 인용할 준비 작업은 끝났다. 서론에서 필요한 부분에 이 논문을 인용 표기할 경우 {\textbackslash}cite 라고 입력한 후 \{ \} 안에 해당 논문을 표시하면 된다. 표시하는 방법은 바로 레퍼런스에서 {\textbackslash}bibitem 이후 \{ \} 안에 적었던 것을 넣어주면 된다. 논문 인용 표시가 문장 마지막에 등장할 때는 마침표의 위치는 인용 표시 다음이다. 아래 문장은 논문 인용 표시의 예로 C. Mok의 2010년 논문에서 인용하였다 \cite{Mok10}.
\begin{lstlisting}
Various plasma instabilities have been proposed as playing important roles during the substorm onset process. These include the tearing \cite{Schindler74, Sitnov97, Zelenyi08}, ballooning \cite{Cheng98, Bhattacharjee98, Dobias04, Zhu03, Saito08, Friedrich01}, lower hybrid drift \cite{Shinohara98, Yoon02, Mok06}, Kelvin--Helmholtz \cite{Rostoker84, Dovias06}, and the ion Weibel \cite{Yoon93, Sadovskii01} instabilities.
\end{lstlisting}
위와 같이 입력한 후 컴파일하면 pdf 파일에는 다음과 같이 나타날 것이다.
\begin{quote}
Various plasma instabilities have been proposed as playing important roles during the substorm onset process. These include the tearing \cite{Schindler74, Sitnov97, Zelenyi08}, ballooning \cite{Cheng98, Bhattacharjee98, Dobias04, Zhu03, Saito08, Friedrich01}, lower hybrid drift \cite{Shinohara98, Yoon02, Mok06}, Kelvin--Helmholtz \cite{Rostoker84, Dovias06}, and the ion Weibel \cite{Yoon93, Sadovskii01} instabilities.
\end{quote}
이 때 참고문헌이 번호 순서대로 나오도록 한다. 또한 세 개 이상의 문헌이 연속된 번호로 이어진 경우 자동으로 첫 번호와 마지막 번호가 hyphen으로 연결된 형태로 등장함을 확인할 수 있다.


참고문헌은 다음의 조건들을 만족해야 한다.
\begin{itemize}
\item{저자가 명시되어야 한다.}
\item{검증이 된 내용이어야 한다.}
\item{이미 출판되어 수정이 불가능해야 한다.}
\end{itemize}
전문 논문 저널에 수록된 논문들은 위 조건들을 만족하므로 되도록 논문을 참고문헌으로 삼도록 한다. 웹사이트는 위 조건들을 만족하지 못하므로 참고문헌으로 부적절하다. 또한 누구라도 책을 출판할 수 있으므로 전문 서적을 참고문헌으로 사용하는 경우에는 널리 받아들여지고 인정받는 서적만 사용해야 한다. 사실 전공 서적의 저자는 여러 연구 논문들을 참고로 하여 책을 집필하기 때문에 전공서적에도 참고 문헌(논문)이 명시되어 있다. 이 경우 전공 서적 대신에 책에서 지시하는 논문을 참고문헌으로 삼도록 한다.

\paragraph{BibTeX 사용 방법}
BibTeX 은 \TeX 에서 참고문헌을 쉽게 관리하기 위한 도구이다. 보통의 경우 thebibliography 환경을 사용하여 참고문헌을 넣는데, BibTeX을 사용할 경우 단순히 다음과 같은 두 줄의 코드로 사용할 수 있다. 
\begin{lstlisting}
\bibliographystyle{ieetr}
\bibliography{bibfile}
\end{lstlisting}
위에서 `bibfile'은 단순히 참고문헌 데이터베이스가 기록되어 있는 BibTeX 파일의 이름이다. 또한, `ieetr'은 마치 MLA, APA style 처럼 BibTeX 에서 사용 가능한 bibliography 스타일 중 하나로, 가장 많이 쓰이는 스타일 중 하나이다. 인터넷 검색을 통해 TeXLive에서 기본적으로 제공되는 BibTeX style의 종류에 관해 알아볼 수 있다. 하지만 경기과학고 \TeX 사용자협회에서는 ieetr 스타일 대신 APA 스타일을 선호하였으나, APA 스타일을 완벽히 구현하는 알려진 스타일 파일이 없어서 직접 `mynewapa.bst' 파일을 제작하였다. 만약 당신도 이렇게 BibTeX 스타일을 직접 제작하고 싶다면 command line interface에서 `latex makebst' 를 치면 된다.

BibTeX이 단순한 thebibliography 환경에 비해 갖는 장점들은 다음과 같다.
\begin{itemize}
	\item 참고문헌들이 본문 내의 인용 순으로 {\bf 자동 정렬}된다.
	\item JabRef 와 같은 BibTeX 관리 프로그램을 이용하여 참고문헌들을 효율적으로 관리할 수 있다.
\end{itemize}

BibTeX 을 사용하기 위해서는 `bibfile.bib' 파일에 각각의 논문의 코드를 쌓아놓기만 하면 된다. 인용 방식은 평소와 같이 \textbackslash cite\{Mok10\} 과 같이 하면 되며, BibTeX 코드는 직접 작성할 수도 있으나 쉽게 얻는 방법은 다음과 같다.
\begin{enumerate}
	\item Google Scholar 에서 검색한 결과에서 `인용'을 클릭한다.
	\item APA, MLA style 등이 나온다. 보통 여기에서 텍스트를 얻어오곤 했을 것이다.
	\item 여기에서 BibTeX 코드를 얻고자 한다면, 하단의 `BibTeX' 을 클릭.
	\item 코드가 나온다. Ctrl+A, Ctrl+C로 복사, bibfile에 붙여넣기.
\end{enumerate}
Google Scholar 외에도 doi2bib 와 같이 DOI 만 갖고 있으면 bibtex 파일을 제공하는 사이트도 있으나, Google Scholar 에 비해서는 적은 양의 정보를 제공하는 것으로 보인다. 다음은 동일한 논문에 대한 thebibliography에서 사용할 코드와 BibTeX 코드이다. 어떤 차이점을 갖는지 비교해 보라.
\begin{quote}
	\verb+\+bibitem\{Mok10\}C. Mok, C.-M. Ryu, P. H. Yoon, and A. T. Y. Lui, ``Obliquely propagating electromagnetic drift ion cyclotron instability'', J. Geophys. Res., \{\verb+\+bf 115\}, A04218 (2010).
\end{quote}
\begin{lstlisting}
@article{Mok10,
title={Obliquely propagating electromagnetic drift ion cyclotron instability},
author={Mok, Chinook and Ryu, Chang-Mo and Yoon, PH and Lui, ATY},
journal={Journal of Geophysical Research: Space Physics},
volume={115},
number={A4},
year={2010},
publisher={Wiley Online Library}
}
\end{lstlisting}